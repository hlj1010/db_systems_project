\documentclass[12pt,arial]{article}
\usepackage{setspace}
\usepackage{listings}
\usepackage{tabularx}
\usepackage{graphicx}
\usepackage[left=1in,right=1in,top=1in,bottom=1in]{geometry}
\usepackage[pdftex,pdfpagelabels,bookmarks,hyperindex,hyperfigures]{hyperref}
\usepackage{color}
\usepackage{tikz-er2}
\definecolor{codegreen}{rgb}{0,0.6,0}
\definecolor{codegray}{rgb}{0.5,0.5,0.5}
\definecolor{codepurple}{rgb}{0.58,0,0.82}
\definecolor{backcolour}{rgb}{0.95,0.95,0.92}
 \usetikzlibrary{positioning}
\usetikzlibrary{shadows}

\tikzstyle{every entity} = [top color=white, bottom color=blue!30, 
                          		  draw=blue!50!black!100, drop shadow]
\tikzstyle{every weak entity} = [drop shadow={shadow xshift=.7ex, 
                                shadow yshift=-.7ex}]
\tikzstyle{every attribute} = [top color=white, bottom color=yellow!20, 
                               draw=yellow, node distance=1cm, drop shadow]
\tikzstyle{every multi attribute} = [top color=white, bottom color=yellow!20, 
                               draw=black, node distance=2cm, drop shadow]
\tikzstyle{every relationship} = [top color=white, bottom color=red!20, 
                                  draw=red!50!black!100, drop shadow]
\tikzstyle{every sub} = [top color=white, bottom color=green!20, 
                         draw=green!50!black!100, drop shadow]
\lstdefinestyle{mystyle}{
    backgroundcolor=\color{backcolour},   
    commentstyle=\color{codegreen},
    keywordstyle=\color{magenta},
    numberstyle=\tiny\color{codegray},
    stringstyle=\color{codepurple},
    basicstyle=\footnotesize,
    breakatwhitespace=false,         
    breaklines=true,                 
    captionpos=b,                    
    keepspaces=true,                 
    numbers=left,                    
    numbersep=5pt,                  
    showspaces=false,                
    showstringspaces=false,
    showtabs=false,                  
    tabsize=2
}
 
\lstset{style=mystyle}
\title{CSE 4020/5231 - Database Systems Project}
\author{Lingjing Huang, Shreyas Ugemuge}
\begin{document}
\maketitle
\tableofcontents
\pagebreak
\section*{Problem Statement}
\addcontentsline{toc}{section}{Problem Statement}
Design a database for storing information about medical doctors, their patients, and ilnesses (using sqlite3):
\begin{enumerate}
	\item Draw its E-R diagram.
	\item Give a relational representation of the E-R diagram that allows for 2 functional preserving and lossless join decompositions. Describe its functional dependencies.
	\item Give the two decompositions and prove that they are lossless join and functional preserving.
	\item Give the SQL DDLs for creating the three versions of the database.
	\item Fill the first database with data (at least 100 rows in each table).
	\item Give the SQL queries to copy the data from the first version into its 2 decompositions.
	\item Propose in english 3 queries that require at least 2 table joins each and such that all tables are involved in at least 2 queries.
	\item Propose SQL implementations of the 3 queries on all three versions of the database.
	\item Test the time in ns for exeuting the 3 queries on each database, by running each of them 1000 times.
\end{enumerate}
\pagebreak


\section{ER-Diagram}
\begin{tikzpicture}[node distance=1.5cm, every edge/.style={link}]
% Doctor entity
\node[entity] (doc) {Doctor};
	%id attribute - primary key
  	\node[attribute] (did) [above=0.5 cm of doc] {\key{doctorID}} edge (doc);
  	% department attribute
  	\node[attribute] (ddept) [above right=0.5 cm of doc,xshift=0.2cm] {department} edge (doc);
  	% start date attribute
  	\node[attribute] (dstart) [above left=0.5 cm of doc] {birthdate} edge (doc);
  	% name attribute composite 
  	\node[attribute] (dname) [below=1.5 cm of doc,xshift=1.2cm] {name} edge (doc);
	% address attribute composite
	\node[attribute] (dadress) [below left=0.5cm  of doc,xshift=0.5cm] {address} edge (doc);
	% phone attribute
	\node[attribute] (docphone) [left=0.7cm  of doc,yshift=4] {phone} edge (doc);
	\node[attribute] (docwe) [below right=0.5cm  of doc] {years\_experience} edge (doc);	
% consult relationship
\node[relationship] (treat) [right=3cm of doc] {Consults} edge (doc);
% patient entity
\node[entity] (pat) [right=1cm of treat] {Patient} edge [total] (treat);
	%id attribute - primary key
  	\node[attribute] (patid) [above right=1 cm of pat] {\key{patientID}} edge (pat);
  	% name attribute composite
  		\node[attribute] (patname) [above=0.4 cm of pat,yshift=0.7 cm] {name} edge (pat);
	% address attribute composite
		\node[attribute] (pad) [below left=1cm  of pat,xshift=1cm,yshift=0.2cm] {address} edge (pat);
	% phone attribute composite
	\node[attribute] (pphone) [below=0.3cm  of pat,xshift=01cm] {phone} edge (pat);
% medical record entity
	\node[attribute] (msex) [above left=0.5 cm of pat, yshift = 0.6 cm] {gender} edge (pat);
	\node[attribute] (mbirth) [right =0.2 cm of pat,yshift=0.5cm] {date of birth} edge (pat);
	\node[attribute] (mblood) [right =0.3 cm of pat, yshift = -0.5cm] {allergies} edge (pat);
% suffer relationship
\node[relationship] (suf) [below =5.2cm of pat] {suffers} edge (pat);
% illness entity
\node[entity] (ill) [left = 2 cm of suf] {illness} edge (suf);
	% name
	\node[attribute] (illnm) [above=1cm  of ill] {\key{Illness\_ID}} edge (ill);
	% symptom attribute 
	\node[attribute] (symt) [below=1cm  of ill,xshift=-2cm] {symptoms} edge (ill);
	% symptom attribute 
	\node[attribute] (ssas) [above left=0.5cm  of ill] {Department} edge (ill);
		\node[attribute] (ssad) [above right=0.5cm  of ill] {name} edge (ill);
	% symptom attribute - primary key
	\node[attribute] (eml) [below=1cm  of ill,xshift=2cm] {emergency\_level} edge (ill);
% can treat relationship
\node[relationship] (tr) [below= 5 cm of doc] {can\_treat} edge [total] (doc) edge (ill);
\end{tikzpicture}


\section{Relational Representation and Functional Dependencies}
\subsection{Relations}
$Doctor (\underline{Doctor\_ID}, Doctor\_Name, Phone, Address, Birthday, Experiences Years, Department) $\\
$Patient (\underline{Patient\_ID}, Patient\_Name, Phone, Address, Birthday, \\ Gender, Allergies History, Doctor\_ID, Illness\_ID) $\\
$Illness (\underline{Illness\_ID}, Illness\_Name, Department, Symptoms, Emergency Level) $\\
\subsection{Functional Dependencies}
\subsubsection{Doctor}
$Doctor\_Name, Phone \rightarrow Doctor\_ID, Department,Experience Years $\\
$Doctor\_ID \rightarrow Doctor_Name, Address, Phone, Birthday$
\subsubsection{Patient}
$ Patient\_Name, Phone \rightarrow Patient_ID,Address, Birthday, Allergies History $\\
$Patient\_ID \rightarrow Patient_Name,Doctor\_ID,Illness\_ID,Phone,Gender$
\subsubsection{Illness}
$Illness\_Name \rightarrow Illness_ID, Department, Symptoms$\\
$Illness\_ID \rightarrow Illness\_Name, Emergency Level$


\section{Decompositions}
\subsection{Decomposition 1}
\subsubsection{Decomposing Doctor}
$R1:$ Doctor Contact Info $(Doctor\_Name, Phone, Address, Department)$\\
$R2:$ Doctor Personal Info $(Doctor\_ID, Doctor\_Name, Phone, Department, Birthday)$ \\ \hfil \break
All other relations remain the same as the original schema
\subsubsection{Proof: Lossless Join and Functional Preserving}
First condition holds true as \\Att(R1) $\cup$ Att(R2) \\= (Doctor\_Name, Phone, Address, Department) $\cup$ (Doctor\_ID, Doctor\_Name, Phone, Department, Birthday) \\= (Doctor\_ID, Doctor\_Name, Phone, Address, Birthday, Experiences Years, Department) \\= Att(R).\\ \hfil \break
Second condition holds true as \\Att(R1) $\cap$  Att(R2) \\= (Doctor\_Name, Phone, Address, Department) $\cap$ (Doctor\_ID, Doctor\_Name, Phone, Department, Birthday) \\= (Doctor\_Name, Phone) \\$\neq \phi$\\ \hfil \break
Third condition holds true as \\Att(R1) $\cap$ Att(R2) = (Doctor\_Name, Phone, Department) is a key of R1 (Doctor\_Name, Phone, Address, Department) because of the FD's given\\ \hfil \break
Furthermore, all dependencies of R either can be a part of R1 or R2 or must be derivable from combination of FD’s of R1 and R2.\\ \hfil \break
\subsection{Decomposition 2}
\subsubsection{Decomposing Patient}
R1: Patient Contact Info $(Patient\_Name, Phone, Address,Gender)$\\
R2: Patient Personal Info $(Patient\_Name, Phone, Patient\_ID, \\Birthday, Allergies History, Doctor\_ID, Illness\_ID)$\\ \hfil \break
All other relations remain the same as the original schema
\subsubsection{Proof: Lossless Join and Functional Preserving}
First condition holds true as \\Att(R1) $\cup$ Att(R2) \\= (Patient\_Name, Phone, Address, Gender) $\cup$ (Patient\_Name, Phone, Patient\_ID, Birthday, Allergies History, Doctor\_ID, Illness\_ID) \\= (Patient\_ID, Patient\_Name, Phone, Address, Birthday, Gender, Allergies History, Doctor\_ID, Illness\_ID)\\= Att(R).\\ \hfil \break
Second condition holds true as \\Att(R1) $\cap$Att(R2) \\= (Patient\_Name, Phone, Address,Gender) $\cap$ (Patient\_Name, Phone, Patient\_ID, Birthday, Allergies History, Doctor\_ID, Illness\_ID) \\$\neq \phi$.\\ \hfil \break
Third condition holds true as \\Att(R1) $\cap$ Att(R2) \\= (Patient\_Name, Phone) is a key of \\R1(Patient\_Name, Phone, Address, Patient\_ID) because of the given FD's\\ \hfil \break
Furthermore, all dependencies of R either can be a part of R1 or R2 or must be derivable from combination of FD’s of R1 and R2.


\section{SQL DDLs}
\subsection{Version 1}
This version is the original database as defined in section 2.1.\\
Filename: \url{create_d.ddl}
\begin{lstlisting}[language=SQL]
	CREATE TABLE `Patient` (
	`Patient_ID`	NUMERIC NOT NULL,
	`Patient_Name`	TEXT NOT NULL,
	`Phone`	NUMERIC,
	`Address`	TEXT,
	`Birthday`	NUMERIC,
	`Gender`	TEXT,
	`Allergies_History`	TEXT,
	`Doctor_ID`	NUMERIC NOT NULL,
	`Illness_ID`	NUMERIC,
	PRIMARY KEY(`Patient_ID`)
);
CREATE TABLE IF NOT EXISTS "Doctor" (
	`Doctor_ID`	NUMERIC NOT NULL UNIQUE,
	`Doctor_Name`	TEXT NOT NULL,
	`Phone`	NUMERIC,
	`Address`	TEXT,
	`Birthday`	NUMERIC,
	`Experiences_Year`	INTEGER,
	`Department`	TEXT,
	PRIMARY KEY(`Doctor_ID`)
);
CREATE TABLE `Illness` (
	`Illness_ID`	NUMERIC NOT NULL UNIQUE,
	`Illness_Name`	TEXT NOT NULL UNIQUE,
	`Department`	TEXT,
	`Symptom`	TEXT,
	`Emergency_Level`	INTEGER,
	PRIMARY KEY(`Illness_ID`)
);
\end{lstlisting}
\subsection{Version 2}
This version was created using Decomposition 1 (Section 3.1)\\
Filename: \url{create_d2.ddl}
\begin{lstlisting}[language=SQL]
CREATE TABLE IF NOT EXISTS "Doctor Personal Info" (
	`Doctor_ID`	NUMERIC NOT NULL UNIQUE,
	`Doctor_Name`	TEXT NOT NULL,
	`Phone`	NUMERIC,
	`Department`	TEXT,
	`Birthday`	INTEGER,
	PRIMARY KEY(`Doctor_ID`)
);
CREATE TABLE IF NOT EXISTS "Doctor Contact Info" (
	`Doctor_Name`	TEXT NOT NULL,
	`Phone`	NUMERIC,
	`Address`	TEXT,
	`Department`	TEXT
);
CREATE TABLE IF NOT EXISTS "Illness" (
	`Illness_ID`	NUMERIC NOT NULL UNIQUE,
	`Illness_Name`	TEXT NOT NULL,
	`Department`	TEXT,
	`Symptom`	TEXT,
	`Emergency_Level`	INTEGER,
	PRIMARY KEY(`Illness_ID`)
);
CREATE TABLE IF NOT EXISTS "Patient" (
	`Patient_ID`	NUMERIC NOT NULL UNIQUE,
	`Patient_Name`	TEXT NOT NULL,
	`Phone`	NUMERIC,
	`Address`	TEXT,
	`Birthday`	NUMERIC,
	`Gender`	TEXT,
	`Allergies_History`	TEXT,
	`Doctor_ID`	NUMERIC,
	`Illness_ID`	NUMERIC,
	PRIMARY KEY(`Patient_ID`)
);
\end{lstlisting}
\subsection{Version 3}
This version was created using Decomposition 2 (Section 3.2)\\
Filename: \url{create_d3.ddl}
\begin{lstlisting}[language=SQL]
CREATE TABLE `Doctor` (
	`Doctor_ID`	NUMERIC NOT NULL UNIQUE,
	`Doctor_Name`	TEXT NOT NULL,
	`Phone`	NUMERIC,
	`Address`	TEXT,
	`Birthday`	NUMERIC,
	`Experiences_Years`	INTEGER,
	`Department`	TEXT,
	PRIMARY KEY(`Doctor_ID`)
);
CREATE TABLE `Illness` (
	`Illness_ID`	NUMERIC NOT NULL UNIQUE,
	`Illness_Name`	TEXT NOT NULL,
	`Department`	TEXT,
	`Symptom`	TEXT,
	`Emergency_Level`	INTEGER,
	PRIMARY KEY(`Illness_ID`)
);
CREATE TABLE IF NOT EXISTS "Patient Personal Info" (
	`Patient_Name`	TEXT NOT NULL,
	`Phone`	NUMERIC,
	`Gender`	TEXT,
	`Birthday`	NUMERIC,
	`Allergies_History`	TEXT,
	`Doctor_ID`	NUMERIC,
	`Illness_ID`	NUMERIC
);
CREATE TABLE IF NOT EXISTS "Patient Contact Info" (
	`Patient_Name`	TEXT NOT NULL,
	`Phone`	NUMERIC,
	`Address`	TEXT,
	`Patient_ID`	INTEGER NOT NULL UNIQUE,
	PRIMARY KEY(`Patient_ID`)
);

\end{lstlisting}


\section{The database with data}


\section{Copy the data from the first version}
\subsection{Copy data to Version 2}
This queries was using to copy data from version 1.
Filename: \url{copy_d_to_d2.sql}
\begin{lstlisting}[language=SQL]
.open d.db
ATTACH 'd2.db' as d2;

INSERT INTO d2.'Doctor Personal Info'
SELECT Doctor_ID, Doctor_Name, Phone, Department, Birthday FROM Doctor;

INSERT INTO d2.'Doctor Contact Info'
SELECT Doctor_Name, Phone, Address, Department FROM Doctor;

INSERT INTO d2.Illness
SELECT * FROM Illness;

INSERT INTO d2.Patient
SELECT * FROM Patient;
\end{lstlisting}
\subsection{Copy data to Version 3}
This queries was using to copy data from version 1.
Filename: \url{copy_d_to_d3.sql}
\begin{lstlisting}[language=SQL]
.open d.db
ATTACH 'd3.db' as d2;

INSERT INTO d2.'Patient Personal Info'
SELECT Patient_Name, Phone, Gender, Birthday,Allergies_History,Doctor_ID,Illness_ID FROM Patient;

INSERT INTO d2.'Patient Contact Info'
SELECT Patient_Name, Phone, Address, Patient_ID FROM Patient;

INSERT INTO d2.Illness
SELECT * FROM Illness;

INSERT INTO d2.Doctor
SELECT * FROM Doctor;
\end{lstlisting}

\section{English Queries}
\subsection{First English Query}
Give Doctor\_Name = 'Taren Batarse', find all his patients Patient\_ID and Patient\_Name.
(Relate to Doctor, Patient table)
\subsection{Second English Query}
Give Patient\_Name is 'Erika Heuberger' and Phone is '2449717107', find Doctor\_Name, and her symptom.
(Relate to Doctor, Illness, Patient table)
\subsection{Third English Query}
Give Illness\_ID = '1', find all patient has this kind of illness Patient\_Name and their Doctor\_Name.
(Relate to Doctor, Illness, Patient table)


\section{SQL implementations of the queries}
\subsection{First English Query}
Give Doctor\_Name = 'Taren Batarse', find all his patients Patient\_ID and Patient\_Name.
(Relate to Doctor, Patient table)
\subsubsection{Version 1}
Filename: \url{query_1_db1.sql}
\begin{lstlisting}[language=SQL]
select p.Patient_ID, p.Patient_Name
from `Patient` as p
join `Doctor` as d on d.Doctor_ID = p.Doctor_ID
where d.Doctor_Name = 'Taren Batarse';
\end{lstlisting}
\subsubsection{Version 2}
Filename: \url{query_1_db2.sql}
\begin{lstlisting}[language=SQL]
select p.Patient_ID , p.Patient_Name
from Patient as p
join `Doctor Personal Info` as d on d.Doctor_ID = p.Doctor_ID
where d.Doctor_Name = 'Taren Batarse';
\end{lstlisting}
\subsubsection{Version 3}
Filename: \url{query_1_db3.sql}
\begin{lstlisting}[language=SQL]
select p.Patient_ID, p.Patient_Name 
from 'Patient Personal Info' as pp
join Doctor as d on d.Doctor_ID = pp.Doctor_ID
join 'Patient Contact info' as p on pp.Patient_Name = p.Patient_Name
where d.Doctor_Name='Taren Batarse';
\end{lstlisting}

\subsection{Second English Query}
Give Patient\_Name is 'Erika Heuberger' and Phone is '2449717107', find Doctor\_Name, and her symptom.
(Relate to Doctor, Illness, Patient table)
\subsubsection{Version 1}
Filename: \url{query_2_db1.sql}
\begin{lstlisting}[language=SQL]
select d.Doctor_Name, i.symptom
from Patient as p
join Doctor as d on d.Doctor_ID = p.Doctor_ID
join Illness as i on i.Illness_ID = p.Illness_ID
where p.Patient_Name = 'Erika Heuberger' and p.Phone = '2449717107'
\end{lstlisting}
\subsubsection{Version 2}
Filename: \url{query_2_db2.sql}
\begin{lstlisting}[language=SQL]
select d.Doctor_Name , i.symptom
from Patient as p
join `Doctor Personal Info` as d on d.Doctor_ID = p.Doctor_ID
join Illness as i on i.Illness_ID = p.Illness_ID
where p.Patient_Name = 'Erika Heuberger' and p.Phone = '2449717107'
\end{lstlisting}
\subsubsection{Version 3}
Filename: \url{query_2_db3.sql}
\begin{lstlisting}[language=SQL]
select d.Doctor_Name , i.Illness_ID
from `Patient Contact Info` as p
join `Patient Personal Info` as pp on p.Phone = pp.Phone
join Doctor as d on pp.Doctor_ID = d.Doctor_ID
join Illness as i on pp.Illness_ID = i.Illness_ID
where p.Patient_Name = 'Erika Heuberger' and p.Phone = '2449717107';
\end{lstlisting}

\subsection{Third English Query}
Give Illness\_ID = '1', find all patient has this kind of illness Patient\_Name and their Doctor\_Name.
(Relate to Doctor, Illness, Patient table)
\subsubsection{Version 1}
Filename: \url{query_3_db1.sql}
\begin{lstlisting}[language=SQL]
select p.Patient_Name, d.Doctor_Name
from Patient as p
join Doctor as d on d.Doctor_ID = p.Doctor_ID
join Illness as i on i.Illness_ID = p.Illness_ID
where p.Illness_ID = '1';
\end{lstlisting}
\subsubsection{Version 2}
Filename: \url{query_3_db2.sql}
\begin{lstlisting}[language=SQL]
select p.Patient_Name , d.Doctor_Name
from Patient as p
join `Doctor Personal Info` as d on d.Doctor_ID = p.Doctor_ID
join Illness as i on i.Illness_ID = p.Illness_ID
where i.Illness_ID = '1';
\end{lstlisting}
\subsubsection{Version 3}
Filename: \url{query_3_db3.sql}
\begin{lstlisting}[language=SQL]
select p.Patient_Name , d.Doctor_Name
from `Patient Personal Info` as pp
join `Patient Contact Info` as p on p.Phone = pp.Phone
join Doctor as d on pp.Doctor_ID = d.Doctor_ID
join Illness as i on i.Illness_ID = pp.Illness_ID
where i.Illness_ID = '1';
\end{lstlisting}


\section{Test the time}

\end{document}